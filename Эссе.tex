\documentclass[12pt]{article}
\usepackage{amsmath}
\usepackage[utf8x]{inputenc}
\usepackage[russian]{babel}
\begin{document}
\begin{center}
{\bf Эссе о карьерных ожиданиях}
\begin{itemize}
\item Почему Политех? \par
Так как я из Петербурга и мне не надо никуда ехать в поисках престижных университетов, я искал ВУЗ у себя дома. Я выбрал Политех, потому что это один из самых престижных технических ВУЗов СПб, дающий качественное образование и хорошие перспективы в дальнейшем трудоустройстве. Также меня привлекла богатая студенческая жизнь (КВН, Студотряды, Адаптеры, Спортивные кружки), которая помогает найти новые связи и расширяет кругозор.

\item Почему Инфокоммуникационные технологии и Cистемы связи? \par
Сами по себе телекоммуникации и системы связи вызывают у меня неподдельный интерес; мне всегда хотелось узнать, как они работает изнутри. Я выбрал инфокоммуникации ещё потому, что в современном мире стремительно растет спрос на сотовые услуги мне бы очень хотелось внести свою лепту в их развитие. 
А ещё направление дает много знаний в сфере радиоэлектроники, программировании, что дает возможность организовать собственный проект.


\item В каких компаниях хотелось бы работать? \par
Мне бы хотелось работать в отрасли, которая неразрывно связана с моим хобби - музыкой.  Native instruments, Pioneer, Gibson, Korg подходят для этого. В этих компаниях меня больше всего интересует работа со схемотехникой (в различном DJ-оборудовании и гитарных процессорах) и создание программ для написания музы~ки. \par
Помимо музыкальных корпораций меня также интересуют операторы мобильной связи (МТС, Tele2, Vodafone). В них меня привлекает работа, связанная с отладкой связи и реализацией через сотовую связь Интернета вещей.

\item Что я жду от университета? \par
Я надеюсь, что университет продвинет меня в области телекоммуникаций, чтобы по окончанию ВУЗа я стал высококвалифицированным специалистом в сфере связи и мог заниматься действительно увлекательной для себя деятельностью.
\end{itemize}
\par
\begin{flushright}
{\bf Зинченко Яков, 13433/3}
\end{flushright}

\end{center}
\end{document}